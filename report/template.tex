\documentclass{vgtc}                          % final (conference style)
\usepackage[T1]{fontenc}
%\documentclass[review]{vgtc}                 % review
%\documentclass[widereview]{vgtc}             % wide-spaced review
%\documentclass[preprint]{vgtc}               % preprint
%\documentclass[electronic]{vgtc}             % electronic version
\usepackage{graphicx}                % allow us to embed graphics files
\DeclareGraphicsExtensions{.pdf,.png,.jpg,.jpeg} % for pdflatex we expect .pdf, .png, or .jpg files
%% it is recomended to use ``\autoref{sec:bla}'' instead of ``Fig.~\ref{sec:bla}''
\graphicspath{{figures/}{pictures/}{images/}{./}} % where to search for the images

\usepackage{hyperref}

\usepackage{microtype}                 % use micro-typography (slightly more compact, better to read)
\PassOptionsToPackage{warn}{textcomp}  % to address font issues with \textrightarrow
\usepackage{textcomp}                  % use better special symbols
\usepackage{mathptmx}                  % use matching math font
\usepackage{times}                     % we use Times as the main font
\renewcommand*\ttdefault{txtt}         % a nicer typewriter font
\usepackage{cite}                      % needed to automatically sort the references
\usepackage{tabu}                      % only used for the table example
\usepackage{booktabs}                  % only used for the table example

%% In preprint mode you may define your own headline.
%\preprinttext{To appear in an IEEE VGTC sponsored conference.}

%% Paper title.

\title{Visualizing Bayesian Optimization for Teaching}

\author{William M. Temple\thanks{e-mail: William.Temple@colorado.edu}\\ %
        \scriptsize University of Colorado Boulder
}%


%% Abstract section.
\abstract{Effectively teaching machine learning, with its high cognitive load,
demands exceptional teaching utilities. Often, students find abstract
discussions or overly-formal mathematical discriptions either difficult to
interpret and understand or insufficiently applicable. In this article, I
explore the application of data visualization techniques to Bayesian
Optimization (Gaussian Processes) and describe the tasks that students engaged
in hypothetical study of this model would likely perform. I further develop
some prototype user-interfaces and address technological challenges essential
to the development of an interactive system for studying Bayesian
Optimization.} % end of abstract

%%%%%%%%%%%%%%%%%%%%%%%%%%%%%%%%%%%%%%%%%%%%%%%%%%%%%%%%%%%%%%%%
%%%%%%%%%%%%%%%%%%%%%% START OF THE PAPER %%%%%%%%%%%%%%%%%%%%%%
%%%%%%%%%%%%%%%%%%%%%%%%%%%%%%%%%%%%%%%%%%%%%%%%%%%%%%%%%%%%%%%%%

\begin{document}

\firstsection{Introduction}

\maketitle

Educational technology systems present interesting design challenges when
analyzed using a data-visualization lens.  Students have different goals
and desires and separate motivations from data analysts. For better or
worse, much of the discourse about data visualization in our coursework
and in the literature we have read for class has focused on the consumption
of data by analysts---or, at least, by users (even untrained or novice users)
\textit{performing} an analytical role.

However, as my own research focuses on the development of educational tools for
novice Computer Science and Programming students, I always try to consider the
ways in which the data visualization techniques that we have discussed apply in
the classroom to enhance the educational experience and provide for a more
robust pedagogy.

Towards this goal, in this project I have focused on building a tool to analyze
the state of a machine learning algorithm. In particular, the system shows the
execution of a Bayesian Optimization using a Gaussian Process Regression. The
system allows a curious user to examine the state of the model and shows not
merely the \textit{result} of the application of the model (which would be the
primary interest of users in analytical roles), but also makes evident the
choices that the model makes during execution and empowers the user to
understand the mathematical reasoning behind those choices.

\subsection{Why Gaussian Processes?}

The Guassian Process is an advanced stochastic process which uses a suite of
functions and strategies for optimizing those functions to create estimations
of the mean and variance over a distribution of other functions. That is to say,
it is relatively complicated.

However, a computer can execute a Gaussian Process in a small series of abstract steps:

\begin{enumerate}

	\item Initialize the Algorithm by sampling a small number of random
		points and fitting an initial regression using a
		\textit{kernel} function.

	\item Use an \textit{acquisition} function to choose a point which is
		likely to improve the estimated mean of the process.

	\item Sample that point and add it to the regression.

	\item Repeat until the model converges on an accurate estimation.
\end{enumerate}

That is to say, the Gaussian Process isn't so complicated that it is immune to
effective decomposition through visualization. It also has certain essential
properties that I believe make it particularly suited to a visual approach to
teaching its concepts. I discuss these properties further in Section 3.

\section{Related Work}

In this section, I discuss some existing literature which informs my design
choices and my thinking about how visualization and task modeling in particular
relate to the goal of education. I also include some discussion of existing
education literature which I feel aids my explanation of why visualization
provides a prefererable environment for learning.

\subsection{Teaching as Presentation}

- Presentation tasks

- Where does teaching fit in to presentation?

\subsection{Task Modeling for Learning}

- Students and teachers have diffeerent motivations

- What tasks do students approach an educational vis with?

- What specific tasks are we going to accomodate?

\subsection{Knowledge Construction}

- Constructivism

- Exploratory design

- User paced

\section{Design}

\subsection{User Experience}

- Coordinated views

- User can examine the result of each iteration

- User sees the coordinated points, showing the alignment of true function, estimation, etc.

\subsection{Technical Challenges}

Early in the prototyping stages of this project, I realized that computational
complexity would pose significant challenges to the interactivity of this
software and its usability. The first Python prototypes of the software could
only compute and render one iteration of the Optimization algorithm about every
25 seconds. Through some code optimization effort, I managed to reduce the
runtime per iteration to 7 seconds, but it became clear during this process
that further optimization would require a shift in development paradigm.

At this latency, the interactivity of the system is threatened, as users may
press buttons that have no effect, and as they begin to use the software, it
takes considerably less than seven seconds to digest the information divulged
by a single iteration of the program. Rudimentary profiling of the code reveals
that the program spends the vast majority (90\%) of its time drawing the four
coordinated plots, as each plot iterates over each pixel in the its image. Even
drawing only one of the plots requires about two seconds using Python, where a
single plot consists of ten thousand points (one hundred points in both the X
and Y directions).

As a proof of concept, in order to solve this problem, I wrote GPU fragment
shader programs using the GPU.js library~\cite{GPUjs}. This library
dyanamically recompiles a restricted subset of JavaScript into GLSL shader
code, which then runs on the SIMD processors in the host machine's Graphics
Processing Unit, resulting in much faster execution. Furthermore, this shader
code deposits resulting textures directly into an HTML \texttt{<canvas>}
element, so that the browser can easily render it. I integrated these fragment
shaders into my concept UI application, and measured that the same machine
could render a function of the same complexity as those in my Python program at
several times the resolution (2048 points in both the X and Y direction) in
about 110 milliseconds~\footnote{Test machine: 6-core Intel i7, 16GB RAM,
NVIDIA GeForce GTX 960}.

\section{Discussion}

\subsection{Prototype Artifacts}

\section{Conclusion}

%\bibliographystyle{abbrv}
\bibliographystyle{abbrv-doi}
%\bibliographystyle{abbrv-doi-narrow}
%\bibliographystyle{abbrv-doi-hyperref}
%\bibliographystyle{abbrv-doi-hyperref-narrow}

\bibliography{template}
\end{document}
